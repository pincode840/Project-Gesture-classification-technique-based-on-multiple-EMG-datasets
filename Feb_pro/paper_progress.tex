\documentclass{article}
\usepackage{graphicx, kotex, xcolor, wrapfig} 
\usepackage{subcaption, float, url, booktabs, amsmath}
\usepackage{listings} 


% 코드 스타일 설정
\lstset{
    language=Python,
    basicstyle=\small\ttfamily,
    keywordstyle=\color{blue},
    commentstyle=\color{gray},
    stringstyle=\color{orange},
    breaklines=true,
    frame=single
}

\title{Project : Gesture classification technique based on multiple EMG datasets}
\author{Name}
\date{\today}

\begin{document}
\maketitle

\section{Database Analysis}
\begin{itemize}
    \item \textbf{Data Loading library:}
\end{itemize}

\begin{lstlisting}
import scipy.io as sio
import matplotlib.pyplot as plt
import numpy as np
import os
import pandas as pd
import csv
import sys
import h5py
\end{lstlisting}



\newpage
\subsection{Ninapro Dataset}
\begin{itemize}
    \item \textbf{Name:} Ninapro (DB5)
    \item \textbf{Data Type:} 16 EMG channels, sampling rate 200Hz, 18 exercises
    \item \textbf{Data shape of one file:} 179,901 rows $\times$ 16 cols
    \item \textbf{Exercises in one file:} 17
    \item \textbf{Data Loading Code:}
\end{itemize}

% ninapro dataset 불러오기.
\begin{lstlisting}
ninapro_df = pd.DataFrame()
for i in range (1,11):
    adress = f"ninapro_db5/ninapro_db_{i}/S{i}_E2_A1"
    filename = adress
    mat = sio.loadmat(filename)
    emg = mat['emg']
    Restimulus = mat['restimulus']
    rerepetition = mat['rerepetition']
    df_emg = pd.DataFrame(emg)
    df_Restimulus = pd.DataFrame(Restimulus)
    df_rerepetition = pd.DataFrame(rerepetition)
    df = pd.concat([df_emg, df_Restimulus], axis=1)
    df = pd.concat([df, df_rerepetition], axis=1)
    df.columns = ['emg1', 'emg2', 'emg3', 'emg4', 'emg5', 'emg6', 'emg7', 'emg8', 'emg9', 'emg10', 'emg11', 'emg12', 'emg13', 'emg14', 'emg15', 'emg16', 'Restimulus', 'rerepetition']
    ninapro_df = pd.concat([ninapro_df, df])
\end{lstlisting}








\newpage
\subsection{MoveR/Nature Dataset}
\begin{itemize}
    \item \textbf{Name:} Nature
    \item \textbf{Data Type:} 16 EMG channels, sampling rate 2000Hz, 6 exercises
    \item \textbf{Exercise Mapping:} 1=6, 2=18, 3=7, 4=5, 5=19, 6=0
    \item \textbf{Data shape of one file:} 16 rows $\times$ 8,000 cols
    \item \textbf{Exercises in one file:} 1
\end{itemize}

% nature dataset 불러오기.
\begin{lstlisting}
grasp_mapping = {
    1: 6,
    2: 18,
    3: 7,
    4: 5,
    5: 19,
    6: 0
}

nature_df = pd.DataFrame()
for i in range(1, 9):
    for j in range(1,3):
        for k in range(1,3):
            filename = fr'nature_data\data\participant_{i}\participant{i}_day{j}_block{k}\emg_data.hdf5'
            data_parame = pd.read_csv(fr'nature_data\data\participant_{i}\participant{i}_day{j}_block{k}\trials.csv') 
            nature_data = h5py.File(filename, 'r')
            data_parame['grasp'] = data_parame['grasp'].replace(grasp_mapping)
            for l in range(0, 150):
                df = pd.DataFrame(np.array(nature_data[f"{l}"]))
                df=df.transpose()
                df['Restimulus'] = ''
                df['Restimulus'] = data_parame['grasp'].iloc[l]
                nature_df = pd.concat([nature_df, df], axis=0)
                print(f"nature_df{i}{j}{k}{l} finished")
nature_df.columns = ['emg1', 'emg2', 'emg3', 'emg4', 'emg5', 'emg6', 'emg7', 'emg8', 'emg9', 'emg10', 'emg11', 'emg12', 'emg13', 'emg14', 'emg15', 'emg16', 'Restimulus']
# nature_df.to_csv('nature_df.csv', index=False)
\end{lstlisting}









\newpage
\subsection{Kaggle Dataset}
\begin{itemize}
    \item \textbf{Name:} Kaggle (EMG Gesture Dataset)
    \item \textbf{Data Type:} 8 EMG channels, sampling rate 985Hz, 12 exercises
    \item \textbf{Data shape of one file:} 8 rows $\times$ 9,980 cols
    \item \textbf{Exercises in one file:} 1
\end{itemize}

\begin{table}[H]
\centering
\small
\begin{tabular}{llc | llc}
\toprule
Abbr. & Full Name & ID & Abbr. & Full Name & ID \\
\midrule
TU  & Thumb Up        & 1  & HO  & Hand Open        & 5  \\
IDX & Index           & 7  & WE  & Wrist Extension  & 14 \\
RA  & Right Angle     & 20 & WF  & Wrist Flexion    & 13 \\
PCE & Peace           & 2  & UD  & Ulnar Deviation  & 16 \\
IL  & Index Little    & 21 & RD  & Radial Deviation & 15 \\
HC  & Hand Close      & 6  &     &                  &    \\
\bottomrule
\end{tabular}
\caption{Kaggle Dataset Gesture Mapping}
\end{table}



\newpage
\section{Channel-wise Encoder + Attention Pooling (CWE-AP)}
\subsection{Mathematical Formulation}





\end{document}